 \documentclass[11pt]{book}
\usepackage{geometry}                % See geo \documentclass[11pt]{article}
\usepackage{geometry}                % See geometry.pdf to learn the layout options. There are lots.
\geometry{letterpaper}                   % ... or a4paper or a5paper or ... 
%\geometry{landscape}                % Activate for for rotated page geometry
%\usepackage[parfill]{parskip}    % Activate to begin paragraphs with an empty line rather than an indent
\usepackage{graphicx}
\usepackage{bbm}
\usepackage{color, soul}
\usepackage{indentfirst}
\usepackage{hyperref}
\usepackage{amsthm}
\usepackage{amsmath}
\usepackage{amssymb}
\usepackage{epstopdf}
\usepackage{multicol}
\usepackage{tikz}
\usepackage{epigraph}
%%%%https://tex.stackexchange.com/questions/122848/tikz-the-pretty-boxes-to-frame-the-theorems-lemma-proposition-etc
\usepackage[utf8]{inputenc}
\UseRawInputEncoding %remove this if experiencing typesetting issues
\usepackage[framemethod=tikz]{mdframed}
\usetikzlibrary{calc}
%\usepackage{chngcntr}

\DeclareGraphicsRule{.tif}{png}{.png}{`convert #1 `dirname #1`/`basename #1 .tif`.png}
%%%%%%%%%%%%%%%%%
\newcommand{\R}{\mathbb{R}}
\newcommand{\N}{\mathbb{N}}
\newcommand{\Q}{\mathbb{Q}}
\newcommand{\Z}{\mathbb{Z}}
\newcommand{\isom}{\cong}
\newcommand{\setm}{\smallsetminus}
\newcommand{\abs}[1]{\lvert#1\rvert}
\newcommand{\tri}{\triangle}

\newcommand{\bfi}{\mathbf i}
\newcommand{\bfj}{\mathbf j}
\newcommand{\bfk}{\mathbf k}
\newcommand{\br}{\mathbf r}
\newcommand{\bv}{\mathbf v}
\newcommand{\bu}{\mathbf u}
\newcommand{\bw}{\mathbf w}
\newcommand{\bF}{\mathbf F}

\newcommand{\va}{\vec{a}}
\newcommand{\vb}{\vec{b}}
\newcommand{\vc}{\vec{c}}
\newcommand{\ve}{\vec{e}}
\newcommand{\vi}{\vec{i}}
\newcommand{\vj}{\vec{j}}
\newcommand{\vk}{\vec{k}}
\newcommand{\vx}{\vec{x}}
\newcommand{\vy}{\vec{y}}
\newcommand{\vv}{\vec{v}}
\newcommand{\vw}{\vec{w}}
\newcommand{\vu}{\vec{u}}
\newcommand{\vr}{\vec{r}}
\newcommand{\vF}{\vec{F}}
\newcommand{\vzero}{\vec{0}}

\newcommand{\horline}{\noindent\rule{14.25cm}{0.6pt}\\}
% counters
\newcounter{theorem}
\newcounter{lemma}
\newcounter{definition}
\newcounter{example}
\newcounter{corollary}

\counterwithin{theorem}{section}
\counterwithin{lemma}{section}
\counterwithin{definition}{section}
\counterwithin{example}{section}
\counterwithin{corollary}{section}

% names for the structures
\newcommand\theoname{Theorem}
\newcommand\lemmname{Lemma}
\newcommand\defname{Definition}
\newcommand\exname{Example}
\newcommand\corname{Corollary}

\makeatletter
% mdf key for the eventual notes in the structures
\def\mdf@@mynote{}
\define@key{mdf}{mynote}{\def\mdf@@mynote{#1}}


% style for theorems
\mdfdefinestyle{mytheo}{
settings={\refstepcounter{theorem}},
linewidth=1pt,
innertopmargin=1.5\baselineskip,
roundcorner=10pt,
backgroundcolor=blue!05,
linecolor=orange,
singleextra={
  \node[xshift=10pt,thick,draw=blue,fill=blue!20,rounded corners,anchor=west] at (P-|O) %
  {\strut{\bfseries\theoname~\thetheorem}\ifdefempty{\mdf@@mynote}{}{~(\mdf@@mynote)}};
},
firstextra={
  \node[xshift=10pt,thick,draw=blue,fill=blue!20,rounded corners,anchor=west] at (P-|O) %
  {\strut{\bfseries\theoname~\thetheorem}\ifdefempty{\mdf@@mynote}{}{~(\mdf@@mynote)}};
}
}

% style for lemmas
\mdfdefinestyle{mylemm}{
settings={\refstepcounter{lemma}},
linewidth=1pt,
innertopmargin=1.5\baselineskip,
roundcorner=10pt,
backgroundcolor=red!05,
linecolor=red!70!black,
singleextra={
  \path let \p1=(P), \p2=(O) in
  node[thick,draw=green!40!black,fill=green!20,rounded corners] at (P-|0.5*\x2+0.5*\x1,0) %
  {\strut{\bfseries\lemmname~\thelemma}\ifdefempty{\mdf@@mynote}{}{~(\mdf@@mynote)}};
},
firstextra={
  \path let \p1=(P), \p2=(O) in
  node[thick,draw=green!40!black,fill=green!20,rounded corners] at (P-|0.5*\x2+0.5*\x1,0) %
  {\strut{\bfseries\lemmname~\thelemma}\ifdefempty{\mdf@@mynote}{}{~(\mdf@@mynote)}};
}
}

% style for definitions 
\mdfdefinestyle{mydef}{
settings={\refstepcounter{definition}},
linewidth=1pt,
innertopmargin=1.5\baselineskip,
roundcorner=10pt,
backgroundcolor=orange!15,
linecolor=red!70!black,
singleextra={
  \path let \p1=(P), \p2=(O) in
  node[xshift = 10pt, thick,draw=green!40!black,fill=orange!5,rounded corners,anchor=west] at (P-|O) %
  {\strut{\bfseries\defname~\thedefinition}\ifdefempty{\mdf@@mynote}{}{~(\mdf@@mynote)}};
},
firstextra={
  \path let \p1=(P), \p2=(O) in
  node[thick,draw=orange!15!black,fill=orange!15,rounded corners, anchor=west] at (P-|0) % 
  {\strut{\bfseries\defname~\thedefinition}\ifdefempty{\mdf@@mynote}{}{~(\mdf@@mynote)}};
}
}

% style for examples
\mdfdefinestyle{myex}{
settings={\refstepcounter{example}},
linewidth=1pt,
innertopmargin=1.5\baselineskip,
roundcorner=10pt,
backgroundcolor=white!15,
linecolor=blue!70!black,
singleextra={
  \path let \p1=(P), \p2=(O) in
  node[xshift = 10pt, thick,draw=blue!40!black,fill=blue!5,rounded corners] at (P-|0.5*\x2+0.5*\x1,0) %
  {\strut{\bfseries\exname~\theexample}\ifdefempty{\mdf@@mynote}{}{~(\mdf@@mynote)}};
},
firstextra={
  \path let \p1=(P), \p2=(O) in
  node[thick,draw=blue!15!black,fill=white!15] at (P-|0.5*\x2+0.5*\x1,0) % 
  {\strut{\bfseries\exname~\theexample}\ifdefempty{\mdf@@mynote}{}{~(\mdf@@mynote)}};
}
}

% style for examples
\mdfdefinestyle{mycor}{
settings={\refstepcounter{corollary}},
linewidth=1pt,
innertopmargin=1.5\baselineskip,
roundcorner=10pt,
backgroundcolor=white!15,
linecolor=blue!70!black,
singleextra={
  \path let \p1=(P), \p2=(O) in
  node[xshift = 10pt, thick,draw=blue!40!black,fill=blue!5,rounded corners] at (P-|0.5*\x2+0.5*\x1,0) %
  {\strut{\bfseries\corname~\thecorollary}\ifdefempty{\mdf@@mynote}{}{~(\mdf@@mynote)}};
},
firstextra={
  \path let \p1=(P), \p2=(O) in
  node[thick,draw=blue!15!black,fill=white!15] at (P-|0.5*\x2+0.5*\x1,0) % 
  {\strut{\bfseries\corname~\thecorollary}\ifdefempty{\mdf@@mynote}{}{~(\mdf@@mynote)}};
}
}

% some auxiliary environments
\newmdenv[style=mytheo]{theor}
\newmdenv[style=mylemm]{lemm}
\newmdenv[style = mydef]{defin}
\newmdenv[style = myex]{examp}
\newmdenv[style = myprf]{prf}
\newmdenv[style = mycor]{cor}
% the actual environments
\newenvironment{theorem}[1][]
	{\begin{theor}[mynote=#1]}
	{\end{theor}}
\newenvironment{lemma}[1][]
	{\begin{lemm}[mynote=#1]}
	{\end{lemm}}
\newenvironment{definition}[1][]
	{\begin{dein}[mynote=#1]}
	{\end{defin}}
\newenvironment{example}[1][]
	{\begin{examp}[mynote=#1]}
	{\end{examp}}
\newenvironment{corollary}[1][]
	{\begin{cor}[mynote=#1]}
	{\end{cor}}

\DeclareMathOperator*{\argmax}{arg\,max}
\DeclareMathOperator*{\argmin}{arg\,min}
\DeclareMathOperator{\proj}{proj}
\DeclareMathOperator{\rref}{rref}
\DeclareMathOperator{\spn}{span}
\DeclareMathOperator{\im}{im}
\newcommand{\rank}{\text{rank}}

\newcommand{\bb}{\begin{bmatrix}}
\newcommand{\eb}{\end{bmatrix}}
\newcommand{\ben}{\begin{enumerate}}
\newcommand{\een}{\end{enumerate}}
\newcommand{\QED}{\begin{flushright}$\blacksquare$\end{flushright}}

%%%%%%%%%%%%%%%epigrapghs

\usepackage{etoolbox}
\makeatletter
\newlength\epitextskip
\pretocmd{\@epitext}{\em}{}{}
\apptocmd{\@epitext}{\em}{}{}
\patchcmd{\epigraph}{\@epitext{#1}\\}{\@epitext{#1}\\[\epitextskip]}{}{}
\makeatother

\setlength\epigraphrule{0pt}
\setlength\epitextskip{2ex}
\setlength\epigraphwidth{.8\textwidth}

\newcommand{\ds}{\displaystyle}
%%%%%%%%%%%%%%%%%


\title{{\bf \href{https://www.overleaf.com/read/gyqrjdfnjyvb}{The Compendium}}}
\author{Alexander Caines}
\date{0.0.4}
\begin{document}
\maketitle

\newpage
\epigraph{Perhaps my old age and fearfulness deceive me, but I suspect that the human species---the unique species---is
about to be extinguished, but the Library will endure\dots}{\href[pdfnewwindow=true]{https://en.wikipedia.org/wiki/Jorge_Luis_Borges_and_mathematics}{Jorge Luis Borges}, \href{https://ia800607.us.archive.org/5/items/TheLibraryOfBabel/babel.pdf}{\emph{The Library of Babel}}}
\newpage
%%%%%%%%%%%%%%%%%%%%%%%%%%%%%%%%%%%%%%%%%%%%%%%%%%%%%%%%%%%%%%%%%%%%%%%%%%%%%%%%%%%%%%%%%%%%%%%%%%%%%%%%%%%%%%%%%%%%%%%%%%%%%%
\section*{Introduction}

During a 1985 lecture \href{https://en.wikipedia.org/wiki/Paul_Erd\%C5\%91s}{Paul Erd\H{o}s} stated ``You don't have to believe in God, but you should
    believe in \emph{The Book}.'' Many colleagues of this brilliant and uncouth mathematician mention
    his murmurings of \emph{The Book}: God's\footnote{If one is given\dots} collection of the most efficient,
    and thus beautiful proofs for each mathematical theorem. 

    Though the mathematical community has not gone to
    the extent of compiling the proofs for every known theorem, there have been attempts. \href{https://proofsfromthebook.github.io/}{\emph{Proofs from THE BOOK}},
    the 1998 book by \href{https://www.quantamagazine.org/gunter-ziegler-and-martin-aigner-seek-gods-perfect-math-proofs-20180319/}{Martin Aigner and G{\"u}nter Ziegler},
    a collection of the most elegant proofs from the fields of
    number theory, geometry, analysis, combinatorics, and graph theory. Recently, Evan Chen---a retired math Olympian and
    current\footnote{as of 2020} math PhD at MIT---has created an \emph{exposition} titled
    \href{https://web.evanchen.cc/napkin.html}{The Napkin}. The (infinite) Napkin is his attempt to compile the most 
    essential statements one needs to understand the beauty and importance of mathematics. It features overviews of
    a myriad of different subjects studied by most undergraduate mathematicians. The Napkin provides the curious a non-trivial
    foundation of their subject of interest; succeeding in honing their mathematical abilities and providing a gateway to 
    more profound investigations.

    Additionally, over the past decade, communities of open-source programmers have created multiple chrestomathy websites
    such as \href{http://rosettacode.org/wiki/Rosetta_Code}{Rosetta Code} and the 
    \href{https://www.algorithm-archive.org/}{Algorithm archive} that 
    aggregate elegant computer programs for different algorithms.
    Given the Curry-Howard correspondence---a relationship between mathematical proofs and computer programs---I 
    thought that this would be worth mentioning. 
    

    All, in all, \emph{that} is what this document tries to be (though with a far more restricted scope and much less precision).
    I hope to compile every theorem, lemma, corollary,
and definition covered over the course of MATH-311. There will be a table of
contents which indexes the statements as they appear in \textbf{\href{https://www.academia.edu/35915359/From_Calculus_to_Analysis_-_Steen_Pedersen_Springer_}{From Calculus to Analysis}}\footnote{Register---\emph{for free}---and it is yours.}.
Definitions, in their totality, will be enumerated first and will be subsequently followed by 
the theorems that we have proved or covered thus far.
Any additional propositions that would otherwise be in the compendium will be 
added at the appropriate time and made note of. Feel free to edit and comment by footnote.
This document serves as a palimpsest for our study of real analysis this term.

	Finally, this document is meant to be pedagogical. So I will attempt to provide commentary and references to helpful,
	auxiliary resources when I can. Have fun!

	P.S. \dots \textcolor{red}{\hl{Do note, that if you want to see the hyper-links in this document, you must download the pdf!}} Especially in this introduction, I have linked to many extra-curricular
sources, which I enjoy visiting or enjoyed having visited. If you have any suggestions, don't be afraid to email me
at \href{mailto:cainesa21@mail.wlu.edu}{cainesa21@mail.wlu.edu} or \href{mailto:alecaines@gmail.com}{alecaines@gmail.com}.
\newpage

\section*{Versioning}
The most critical part of this project is its capacity as an open source document. Given the assumption that most of the my peers are not acquainted with 
\href{https://github.com/features#team-management}{GitHub}, a more rudimentary version control system must be developed in order to maintain current and alternative 
versions of \emph{The Compendium}. Because of the nature of this project, future versions of \emph{The Compendium} will not see great structural changes. Most changes
to the document will entail the addition or editing of mathematical statements. Because new definitions are anticipated to be added through the rest of the term (the 
release date is 4/2/2020), we will avoid a \href{https://en.wikipedia.org/wiki/Unary_numeral_system}{unary versioning} system as employed by the TeX typesetting system.

\subsection*{Versioning Schema}
I propose the following schema: A major version change will entail a change in structure of the document. A minor version change will entail the correction a previously
invalid proof or definition, as well as spelling errors. Patch version changes will entail the addition of new definitions and theorems. If the number of patches explodes
in the near future, I will implement a modulo on the patch version and record the multiplicity of patches in an appended digit to the version number.

\subsection*{Submitting Changes}
Each individual who intends to edit their copy of \emph{The Compendium} shall make the edits to the \LaTeX document and resubmit the TeX file to me via email. In the 
email, make note of 
	\begin{center}
		\begin{enumerate}
			\item[1.] The mathematical statements (theorems, proofs, definitions, etc\dots) changed\\
			\item[2.] The line numbers at which the changes took place.\\
			\item[3.] A brief description (need not be elaborate at all) of the changes.
		\end{enumerate}
	\end{center}
	Given that I believe the changes are appropriate, I will then alter the version of \emph{The Compendium} and its state. I will then release the \href{https://www.overleaf.com/read/gyqrjdfnjyvb}{newest version} on 
	Overleaf, for all to access. All version changes will be recorded at after the section titled \textbf{Theorems, Lemmas, Corollaries, and Proofs}. Though, do note,
	no one is stopping you from hosting your own version! If you would like to release your own version of \emph{The Compendium}, then reset the version to $0.0.0$,
	label yourself as the co-author on the title, and \LaTeX away!

%%%%%%%%%%%%%%%%%%%%%%%%%%%%%%%%%%%%%%%%%%%%%%%%%%%%%%%%%%%%%%%%%%%%%%%%%%%%%%%%%%%%%%%%%%%%%%%%%%%%%%%%%%%%%%%%%%%%%%%%%%%%%
\newpage
\tableofcontents
\newpage
\addtocontents{toc}{\setcounter{tocdepth}{3}}
\section{Definitions}
	\subsection{Infinite Decimal}
		\begin{defin}
			An \textbf{infinite decimal} is an expression of the form $\pm d_0.d_1.d_2\dots,$ where
			$d_0 \in \mathbb{N}_0$ and $d_k \in \{0,1,2,3,4,5,6,7,8,9\}$ for $k \geq 1$.
		\end{defin}
	\subsubsection{Repeating infinite decimal}
		\begin{defin}
			An \emph{infinite decimal} is \textbf{repeating} if there are $k,m \in \N \ni d_{j+m} = d_j, \forall j \geq k$
		\end{defin}
		\begin{examp}
			1.67234523452345\dots := 1.6$\overline{2345}$,
			where $k = 3, m = 4,$ and \\$j := \text{ any arbitrary natural number greater than $k$}$.
		\end{examp}
	\subsection{Finite Decimal}
		\begin{defin}
			A \textbf{finite decimal} is a decimal of the form:
			$d_0.d_1d_2\dots d_n := d_0.d_1d_2\dots \overline{0}$
		\end{defin}
	\subsection{Irrational number}
		\begin{defin}
			An \textbf{irrational number} is a real number that is not rational.
		\end{defin}
		\begin{examp}
			Let $x := 0.101001000100001\dots$.Note that there is no consistent repeating string of digits in $x$. 
			So, $x$ is irrational.
		\end{examp}

	\subsection{Density}
		\begin{defin}
			Let $A$ and $B$ be sets. We say $A$ is \textbf{dense} in $B$ if any interval centered at a point in $B$
			contains at least one point from $A$.
			In discrete terms, $\forall b \in B, \forall r > 0, \exists a \in A \ni a \in (b - r, b + r)$\footnote{Otherwise written $|a-b| < r$.}.
		\end{defin}
	\subsection{Accumulation Point}
		\begin{defin}
			Let $D \subseteq \R$ and $a \in \R$. $a$ is an \textbf{accumulation point} of $D$ if $\forall \varepsilon > 0,
			\exists x \in D \ni 0 < |x - a | < \varepsilon$.
		\end{defin}
		\begin{examp}
			\textbf{Claim:} 0 is an accumulation point of $(-1,1)$.\hfill\break
			\textbf{Proof:} Let $\varepsilon > 0$ be given. Suppose $\varepsilon \leq 1$. Let $x := \frac{\varepsilon}{2}$.
			Then, $0 < x = |x - 0| < 1$. So, $x \in (-1,1)$. Note, $x = \frac{\varepsilon}{2} < \varepsilon$. So, 
			$0 < |x - 0| < \varepsilon$. \QED
			\horline
			\textbf{Comments:} The above proves to be an interesting result because given any $m > 0$, if $0 < r \leq m$, 
			then $\exists x \in D \ni 0 < |x - a| < r$. Consequently, $a$ must be an accumulation point of $D$.
		\end{examp}
	\subsection{Convergence}
		\begin{defin}
			Let $D \subseteq \R$, $a,L \in \R$, $f: D \to \R$, and $a$ be an accumulation point of $D$. We say $f$
			\textbf{converges} to $L$ as $x$ approaches $a$ in $D$ provided $\forall \varepsilon > 0$, $\exists \delta > 0 \ni
			\forall x \in D, 0 < |x - a| < \delta \implies |f(x) - L| < \varepsilon$.
		\end{defin}
		\begin{examp}
			\textbf{Claim:} $\displaystyle\lim_{x \to 4}{f(x) = 7}$.\\
			\textbf{Proof:} Pick $\varepsilon > 0$ and let $\delta - \frac{\varepsilon}{3}$. Then, $\forall x \in \R$,
				\begin{center}
					\begin{math}
						\begin{aligned}
							|f(x) - 7| & = |3x - 12|\\
							& = 3|x - 4|\\
							& < 3\cdot\bigg(\frac{\varepsilon}{3}\bigg)\\
							& = \varepsilon
						\end{aligned}
					\end{math}
				\end{center}
				\QED
		\end{examp}
	\subsection{Local Behavior}
		\begin{defin}
			Let $D \subseteq \R$ and $a$ an accumulation point in $D$ be given. Additionally, let $f: D \to \R$ where $f$
			has \textbf{local behavior} near $a$ if $\exists \delta > 0 \ni f(x)$ has the property, $\forall x \in (a - \delta, a + \delta)$.
		\end{defin}
	\subsection{Restriction}
		\begin{defin}
			Let $D \subseteq \R$ and $f: D \to \R$. If $E \subseteq D$, we define the \textbf{restriction} of $f$ to $E$ to be the function $f\Big|_{E}$ with domain
			$E$ and having the same values as $f$ on $E$. Formally, $$f\Big|_E:E \to \R, f\Big|_E(x) = f(x), \forall x \in E$$
			Note, $f\Big|_E(x)$ is undefined $\forall x \in D\subseteq E$.
		\end{defin}
	\subsection{Limit at infinity}
		\begin{defin}
			$\displaystyle\lim_{x \to \infty}{f(x) = L}$ is defined when $\forall \varepsilon > 0$, $\exists N \in \R, \forall x \in D, N < x \implies |f(x) - L| <
			\varepsilon$.
		\end{defin}
		\begin{examp}
			\textbf{Claim:} $\displaystyle\lim_{x \to \infty}{\frac{1}{x^2} = 0}$.\hfill\break
			\textbf{Proof:} Let $\varepsilon > 0$ be given. We want to find $N$ such that $x > N \implies \frac{1}{x^2} < \varepsilon$. Note, $x^2 > \frac{1}{\varepsilon}$.
			Let $N := \frac{1}{\sqrt{\varepsilon}}$. Therefore, if $ x > N \Leftrightarrow x > \frac{1}{\sqrt{\varepsilon}}$, then $x^2 > \frac{1}{\varepsilon}$.
			So, $|\frac{1}{x^2} - 0| = \frac{1}{x^2} < \varepsilon$.
			\QED
		\end{examp}
	\subsection{Sequence}
		\begin{defin}
			A \textbf{sequence} of real numbers is a function $x:\N \to \R$ denoted $(x)_{n = a}^b$ where $x_n, \forall n \in [a,b]$ are the element of the 
			sequence at from indices $a$ to $b$.
		\end{defin}
	\subsection{Bounded sequence}		
		\begin{defin}
			A sequence is \textbf{bounded} if there is a $k$ such that $|x_n| \leq k, \forall n$.
		\end{defin}
	\subsection{Null Sequence}
		\begin{defin}
			A sequence $x_n$ is \textbf{null} if given any $\varepsilon > 0$, $\exists N \in \N \ni \forall n \in \N^{>N}, n \implies |x_n| < \varepsilon$.\hfill\break
			\horline
			\textbf{Comment:}
				\begin{enumerate}
					\item[1.] This means that $x_n \to 0$ if $x_n$ is null.
					\item[2.] This also means $x_n \to p$ if and only if $(x_n - p)$ is null.
				\end{enumerate}
			
		\end{defin}
		\newpage
		\begin{examp}
			\textbf{Claim:} $(1 + (-1)^n)$ is not null.\hfill\break
			\textbf{Proof:} Note, if a sequence $(x)$ is not null, then $\exists \varepsilon > 0 \ni \forall N \in \N, \exists n \in \N$ where $n > N$ and $|x_n| \geq
			\varepsilon$. Let $x_n = 1 + (01)^n, \forall n$. So, $|x_n| = |1+(-1)^n| = \begin{cases} 0 & \text{ when} n \text{ is odd.}\\
				1 & \text{ when} n \text{ is even.}\end{cases}$. Choose $\varepsilon := 1$ and $N \in \N$ be arbitrary. Now, choose $n := 2N$. Then, 
				$n > N$ trivially. Additionally, $|1 + (-1)^n| = |1 + (-1)^{2N}| = 2 > 1$.
				\QED
		\end{examp}
	\subsection{Continuous function}
		\begin{defin}
			Let $D \subseteq \R$. A function$f: D \to \R$ is \textbf{continuous} at a point $a \in D$ if $\varepsilon > 0, \exists \delta > 0 \ni \forall x \in D$,
			$|x - a| < \delta \implies |f(x) - f(a)| < \varepsilon$.\hfill\break
			\horline
			\textbf{Comment:} Note, if $a$ is an accumulation point, then $\displaystyle\lim_{x \to a}{f(x) = f(a)}$. If $a$ is not an accumulation point,
			but $f$ is continuous, then $|x - a| < \delta \implies x = a$. Such an $a$ is called an \textbf{isolated point}.
		\end{defin}
	\subsection{Removable Discontinuity}
	\label{subsec:removablediscontinuity}
		\begin{defin}
			Suppose $f$ is discontinuous at $a$. Then, $f$ has a \textbf{removable discontinuity} at $a$ if there is a function $g$ such that $g$ is continuous
			at $a$ and $g(x) = f(x), \forall x \neq a$.
		\end{defin}
		\begin{examp}
			\textbf{Claim:} $f(x) := \frac{x^2 -1}{x-1}$ has a removable discontinuity at $a = 1$.\hfill \break
			\textbf{Proof:} Let $g(x) := x + 1$. Then $g(x) = f(x), \forall x \neq 1$ and $g$ is continuous at $x = 1$. 
			So, $f$ has a removable discontinuity at $x = 1$.

		\end{examp}
	\newpage
	\subsection{Right-Sided Continuity}
	\label{subsec:rightsidedcontinuity}
		\begin{defin}
			Let $D \subseteq \R$, $f: D \to \R$, and $a \in D$. $f$ is \textbf{continuous from the right} at $a$ if $\forall \varepsilon > 0, \exists \delta > 0 \ni
			\forall x \in D, 0 < | x - a| < \delta \implies |f(x) - f(a)| < \varepsilon$.
		\end{defin}
	\subsection{Left-Sided Continuity}
	\label{subsec:leftsidedcontinuity}
		\begin{defin}
			Let $D \subseteq \R$, $f: D \to \R$, and $a \in D$. $f$ is \textbf{continuous from the left} at $a$ if $\forall \varepsilon > 0, \exists \delta > 0 \ni
			\forall x \in D, -\delta < x - a < 0 \implies |f(x) - f(a) | < \varepsilon$.
		\end{defin}
	\subsection{Upper Bound}
	\label{subsec:upperbound}
		\begin{defin}
			Let $A \subseteq \R$. A real number $u \in \R$ is an \textbf{upper bound} for $A$ if each $a \in A \implies a \leq u$.
		\end{defin}
	\subsection{Lower Bound}
	\label{subsec:lowerbound}
		\begin{defin}
			Let $A  \subseteq \R$. A real number $u \in \R$ is a \textbf{lower bound} for $A$ if each $a \in A \implies a \geq u$. 
		\end{defin}
	\subsection{Maximum}
	\label{subsec:maximum}
		\begin{defin}
			Let $A \subseteq \R$. A real number $u \in \R$ is the \textbf{maximum} of $A$ if each $u \in A$ implies $u$ is an \hyperref[subsec:upperbound]{upper bound}
			for $A$.
		\end{defin}
	\subsection{Minimum}
	\label{subsec:minimum}
		\begin{defin}
			Let $A \subseteq \R$. A real number $u \in \R$ is the \textbf{minimum} of $A$ if $u \in A$ implies $u$ is a \hyperref[subsec:lowerbound]{lower bound} for $A$.
		\end{defin}
	\subsection{Supremum or Least Upper Bound}
	\label{subsec:supremum}
		\begin{defin}
			Let $ \subseteq \R$ and $u \in $. $u$ is a \textbf{supremum} or \textbf{least upper bound} if $u$ is an \hyperref[subsec:upperbound]{upper bound}
			of $A$ and no number smaller than $u$ is an upper bound.
		\end{defin}
	\subsection{Infimum or Greatest Lower Bound}
	\label{subsec:infimum}
		\begin{defin}
			Let $A \subseteq \R$ and $u \in \R$. $u$ is an \textbf{infimum} or \textbf{greatest lower bound} if $u$ is a \hyperref[subsec:lowerbound]{lower bound} for
			$A$ and no number larger than $u$ is a lower bound.
		\end{defin}
	\subsection{Intermediate Value Property}
	\label{subsec:intermediatevalueproperty}
		\begin{defin}
			A subset $A \subseteq \R$ has the \textbf{intermediate value property} if $\forall x,y \in A$ and $t \in \R$, $x < t < y \implies t \in A$.
		\end{defin}
%%%%%%%%%%%%%%%%%%%%%%%%%%%%%%%%%%%%%%%%%%%%%%%%%%%%%%%%%%%%%%%%%%%%%%%%%%%%%%%%%%%%%%%%%%%%%%%%%%%%%%%%%%%%%%%%%%%%%%%%%%%%%%%%%%%%%	
%%%%%%%%%%%%%%%%%%%%%%%%%%%%%%%%%%%%%%%%%%%%% Theorems, Lemmas, Corollaries, and Proofs %%%%%%%%%%%%%%%%%%%%%%%%%%%%%%%%%%%%%%%%%%%%%	
%%%%%%%%%%%%%%%%%%%%%%%%%%%%%%%%%%%%%%%%%%%%%%%%%%%%%%%%%%%%%%%%%%%%%%%%%%%%%%%%%%%%%%%%%%%%%%%%%%%%%%%%%%%%%%%%%%%%%%%%%%%%%%%%%%%%%	
\newpage
	\subsection{Countable}
	\label{subsec:countable}
		\begin{defin}
			A set is called \textbf{countable} if its elements can be enumerated by the naturals.
		\end{defin}
	\subsection{Increasing}
	\label{subsec:increasing}
		\begin{defin}
			Let $I$ be some interval. A function $f: I \to \R$ is \textbf{increasing} if $\forall x,y \in I, x < y \implies f(x) \leq f(y)$.
		\end{defin}
	\subsection{Decreasing}
	\label{subsec:decreasing}
		\begin{defin}
			Let $I$ be some interval. A function $f:I \to \R$ is \textbf{decreasing} if $\forall x,y \in I$, $x < y \implies f(x) \geq f(y)$. 
		\end{defin}
	\subsection{Monotone}
	\label{subsec:monotone}
		\begin{defin}
			A function is \textbf{monotone} if it is either increasing or decreasing.
		\end{defin}
	\subsection{Jump Discontinuity}
	\label{subsec:jumpdiscontinuity}
		\begin{defin}
			A \textbf{jump discontinuity} is defined when one-sided limits exist at a point $a$ but are not equal.
		\end{defin}
	\subsection{Bounded Function}
	\label{subsec:bounded}
		\begin{defin}
			A function $f: D \to \R$ is \textbf{bounded} on $E \subseteq D$ if $\exists M \in \R \ni |f(x)| \leq M, \forall x \in E$.
		\end{defin}
	\subsection{Compact Interval}
	\label{subsec:compact}
		\begin{defin}
			Suppose $a<b$ are real. A \textbf{compact interval} $I$ is a closed and bounded interval of the form $I := [a,b]$.
		\end{defin}
	\newpage
	\subsection{Uniform Continuity}
	\label{subsec:uniformcontinuity}
		\begin{defin}
			A function $f:D \to \R$ is \textbf{uniformly continuous} on $D$ if given any $\varepsilon > 0$, there is a $\delta > 0 \ni \forall x,y \in D, |x - y| 
			< \delta \implies |f(x) - f(y)| < \varepsilon$.
		\end{defin}
		\begin{examp}
			\textbf{Claim:} $f(x) := x^2$ is uniformly continuous on $D := \{x \in \R| |x| < 7 \}$.\hfill\break
			\textbf{Proof:} Let $\varepsilon > 0$ be given. Note, $|x^2 - y^2| = |(x-y)||(x+y)|$. By the triangle inequality, $|x+y| \leq |x| + |y|$.
			So, $|(x-y)||(x+y)| < 7 + 7 < 14$. So, let $\delta := \frac{\varepsilon}{14}$. Thus,
				\begin{center}
					\begin{math}
						\begin{aligned}
							|f(x) - f(y)| & = |x-y||x+y|\\
							& = \frac{\varepsilon}{14} \cdot 14\\
							& = 14
						\end{aligned}
					\end{math}
				\end{center}
				\QED
		\end{examp}
	\subsection{Differentiable}
	\label{subsec:differentiable}
		\begin{defin}
			Let $D \subseteq \R$. A function $f: D \to \R$ is \textbf{differentiable} at $a \in D$ if $a$ is an accumulation point of $D$ with derivative $b$ gives,
			$\forall \varepsilon > 0 \ni \forall x \in D, ) < |x -a|< \delta \implies|\frac{f(x) - f(a)}{x-a} - b| < \varepsilon$.
		\end{defin}
	\subsection{Right-hand Derivative}
	\label{subsec:rhderivative}
	\begin{defin} The \textbf{right-hand derivative} $(f^{\prime})^+(a)$ is defined, $(f^{\prime})^+ := \displaystyle\lim_{x \to a^+}{\frac{f(x) - f(a)}{x - a}}$.
	\end{defin}

	\subsection{Left-hand Derivative}
	\label{subsec:lhderivative}
	\begin{defin}
		The \textbf{left-hand derivative} $(f^{\prime})^-(a)$ is defined, $(f^{\prime})^- := \displaystyle\lim_{x \to a^-}{\frac{f(x) - f(a)}{x - a}}$.
	\end{defin}
	\subsection{Local Maximum}
	\label{subsec:localmaximum}
		\begin{defin}
			$f(a)$ is a \textbf{local maximum} of $f$ if $\exists \delta > 0 \ni x \in (a-\delta,a+\delta) \implies f(x) \geq f(a)$.
		\end{defin}
	\subsection{Local Minimum}
	\label{subsec:locaminimum}
		\begin{defin}
			$f(a)$ is a \textbf{local minimum} of $f$ if $\exists \delta > 0 \ni x \in (a-\delta,a+\delta) \implies f(x) \geq f(a)$.
		\end{defin}
	\subsection{Partition}
	\label{subsec:partition}
		\begin{defin}
			A \textbf{partition} of a closed and bounded\footnote{Compact, in other words\dots} interval $[a,b]$ is a finite collection of points,
			$$a = x_0 < x_1 < \dots < x_n = b$$
		\end{defin}

	\subsection{Step Function}
	\label{subsec:stepfunction}
		\begin{defin}
			Given a partition $a = x_0 < \dots < x_n = b$, of $[a,b]$, a \textbf{step function} is a function,
			$$s:[a,b] \to \R, s(x) := \sum_{k = 1}^n{A_k \mathbbm{1}_(x_{k-1},x_k)(x)}$$
			that is constant on $(x_{k-1},x_k)$.
		\end{defin}

	\subsection{Lower Step Function}
	\label{subsec:lowerstepfunction}
		\begin{defin}
			A step function $s:[a,b] \to \R$ is a \textbf{lower step function} for $f$ if 
			$$s(x) \leq f(x), \forall x \in [a,b]$$
			except possibly at partition points.
		\end{defin}

	\subsection{Upper Step Function}
	\label{subsec:upperstepfunction}
		\begin{defin}
			An step function $s:[a,b] \to \R$ is an \textbf{upper step function} if 
			$$f(x) \leq s(x), \forall x \in [a,b]$$
			except possibly at partition points.
		\end{defin}
	
	\subsection{Lower Integral}
	\label{subsec:lowerintegral}
		\begin{defin}
			The \textbf{lower integral} of $f$ $\underline{\int_a^b }f$ is the \hyperref[subsec:supremum]{supremum}  of the set of lower sums for $a$.
			$$\underline{\int_a^b}f := \text{sup}\{\sum s | s \text{ is a lower step function of } f \}$$
		\end{defin}

	\subsection{Upper Integral}
	\label{subsec:upperintegral}
		\begin{defin}
			The \textbf{upper integral} of $f$ $\overline{\int_a^b }f$ is the \hyperref[subsec:infimum]{infimum}  of the set of upper sums for $a$.
			$$\overline{\int_a^b}f := \text{sup}\{\sum s | s \text{ is an upper step function of } f \}$$
		\end{defin}

		\begin{examp}
			\textbf{Claim:} The set of upper sums is nonempty and bounded below. 
			\textbf{Proof:} Let $S(x) = M$. Then, $s(x) \geq f(x), \forall x \in [a,b], \forall x \in [a,b]$. So, $S(x)$ is an upper step function and $\sum S$ is an upper sum. So, 
			the set of upper sums is not nonempty.

			Now, let $S = \sum_{k=1}^n{A_k \mathbbm{1}_{(x_{k-1}, x_k)}}$ be an upper set function. Then $A_k \geq f(x)$. Note, $f(x) \geq M$. So, $A_k \geq M$. Thus, 
			$$\sum S = \sum_{i = 1}^M{A_i(x_i - x_{i-1})} \geq \sum_{i = 1}^N{M(x_i - x_{i-1})} = M(b-a)$$
			\QED
		\end{examp}

	\subsection{Riemann Integrable}
	\label{subsec:riemannintegrable}
		\begin{defin}
			A bounded function $f:[a.b] \to \R$ is \textbf{(Riemann) integrable} if $\underline{\int_a^b}{f} = \underline{\int_a^b}{f}$
		\end{defin}
		\begin{examp}
			Let $s$ be a lower step function for $g$. Then, $s(x) \leq f(x) \leq g(x), \forall x$. Therefore, 
			$$\sup\{\sum s | s \text{ is a lower step function for } g \} \leq \sup\{\sum S | S \text{ is a lower step function for } g \}$$
			So, $\underline{\int_a^b}{f} \leq \underline{\int_a^b}{g}$. Since $f$ and $g$ are integrable, $\int_a^b = \underline{\int_a^b}{f} = \underline{\int_a^b}{g}$.
			\QED
		\end{examp}

	\subsection{Refinement}
	\label{subsec:refinement}
		\begin{defin}
			A partition $Q: a = y_0 < y_1 < \dots < y_n = b$ is a \textbf{refinement} of a partition $P: a = x_0 < \dots < x_m = b$ if 
			$$\{x_j | j = 0,\dots,m \} \subset \{y_k | k = 0,\dots,n\}$$
		\end{defin}

\section{Theorems, Lemmas, Corollaries, and Proofs}
	\subsection{Theorem 1.1.1}
		\begin{theor} 
			The set of finite decimals equals the set of infinite decimals terminating in repeating nines.
		\end{theor}
	\subsection{Theorem 1.1.2}
		\begin{theor}
			An \textbf{infinite decimal} is a \emph{rational number} if and only if it is repeating.
		\end{theor}

		\begin{examp}
			\textbf{Claim:} 0.$\overline{234} \in \Q$.\hfill\break
			\textbf{Proof:} Let $x = 0.\overline{234}$. Then, $1000x = 234.\overline{234}$.
			\begin{center}
				\begin{math}
					\begin{aligned}
				1000x & = 234 + x\\
				x & = \frac{234}{999}
					\end{aligned}	
				\end{math}
			\end{center}
			So, $x \in \Q$. \QED
		\end{examp}

		\begin{examp}
			\textbf{Claim:} If $\frac{p}{q}$ is rational, then the repeating part of its
			infinite decimal has a maximum length of $q-1$.\hfill\break
			\textbf{Proof:} Without loss of generality, consider $\frac{1}{7} \in \Q$. We want
			to find the infinite decimal form of $\frac{1}{7}$. Assuming long-divison works,
			one can discover find that $\frac{1}{7} = 0.\overline{142857}$.\hfill\break
			Note, there are seven possible elements in the set of remainders, $\Z_7$. 
			If 0 is the remainder, then the decimal terminates. So, the maximum possible length
			of the repeating decimal for $\frac{1}{7}$ is six. \QED
		\end{examp}

	\subsection{Lemma 1.1.4 (Archimedean property)}
	\label{subsec:lemma114}
		\begin{lemm}
			If $r > 0$, $\exists n \in \mathbb{Z}^{>0} \ni r > \frac{1}{10^N}$.

			``\emph{Given any number $r$, there exists a rational $\frac{1}{10^N}$ smaller than it.}''\hfill\break
			\horline
			\textbf{Proof:} Let $r$ be a decimal $d_0.d_1d_2\dots > 0$. Since $r \neq 0$,\hfill\break
			$\exists n \in \N \ni d_n \neq 0$. Note, $\frac{1}{10^n} = 0.00\dots d_n$ where\hfill\break
			$d_n = 1$. Additionally, note $r > \frac{1}{10^{n+1}}$. \QED

		\end{lemm}
	\subsection{Corollary 1.1.5}
	\label{subsec:cor115}
		\begin{cor}
			If $r \geq 0$ and $r < \frac{1}{10^n}, \forall n \in \N$, then $r = 0$.
		\end{cor}
	\subsection{Theorem 1.1.6 (Density of the Rationals in the Reals)}
	\label{subsec:theorem116}
		\begin{theor}
			$\Q$ is dense in $\R$.\hfill\break
			\horline
			\textbf{Proof:} Let $r>0$ and $x \in \R$ be given. Note, $x$ can be expressed as an infinite
			decimal $x := d_0.d_1d_2\dots$. By \hyperref[subsec:lemma114]{\textbf{Lemma 1.1.4}}, $\exists n \in \Z^{>0} \ni r > \frac{1}{10^n}$.
			Let $y,z \in \Q \ni$ $y := d_0.d_1\dots \overline{0}$ and $z := d_0.d_1\dots d_n \overline{0}$. Note,
			if $x > 0$, $y \leq x \leq z$. If $x < 0$, $z < x < y$.\hfill\break
			Now, suppose $x > 0$. So, $y \leq x \leq z$. Note, $z - y = \frac{1}{10^n} \Leftrightarrow z = y + \frac{1}{10^n}$.\\
			So, the following,
			$$ x-r < x \leq y + \frac{1}{10^n} \leq x + \frac{1}{10^n} < r + x$$
			Note, $z = y + \frac{1}{10^n}$. So, $x - r < z < x+r$. Additionally, note $z \in \Q$.\hfill\break
			Now, let $x < 0$. So, $z \leq x \leq y$. Note, $y = z + \frac{1}{10^n}$. The following,
			$$x - r < x \leq z + \frac{1}{10^n} \leq x + \frac{1}{10^n} < r + x$$
			So, $x - r < y < x + r$. Note, $y \in \Q$. Finally, let $x = 0$. $0 \in \Q$ and $0 \in (r-x,x+r), \forall r > 0$.
			So, $\Q$ is dense in $\R$. \QED
		\end{theor}
	\subsection{Theorem 1.1.8 (Density of Irrationals in the Reals)}
	\label{subsec:theorem118}
		\begin{theor}
			Any open interval contains an irrational number.
		\end{theor}
	\subsection{Theorem 1.3.12 (Local boundedness)}
	\label{subsec:theor1312}
		\begin{theor}
			Let $f: D \to \R$, $D \subseteq \R$, and $a$ an accumulation point of $D$ be given. If $\displaystyle\lim_{x \to a}{f(x) = L}$
			exists, then $\exists \delta > 0, \exists M > 0 \ni \forall x \in D, if 0 < |x - a| < \delta$. Consequently, 
			$|f(x)| < M$.\\
			\horline
			\textbf{Proof:} Since $\displaystyle\lim_{x \to a}{f(x) = L}$, $\forall \varepsilon > 0$ (and in particular, $\varepsilon = 1$), $\exists \delta > 0 \ni
			\forall x \in D, 0 < | x - a| \implies |f(x) - L| < 1$. Let $M = 1 + |L|$. Note,
			\begin{center}
				\begin{math}
					\begin{aligned}
						|f(x)| &  = |f(x) - L + L|\\
						& \leq |f(x) - L| + |L|\\
						& < 1+ |L|\\
						& = M
					\end{aligned}
				\end{math}
			\end{center}
			\QED
		\end{theor}
	\newpage
	\subsection{Lemma 1.4.2 (Constant multiple rule)}\footnote{Don't ask me why they structured it like this\dots}
		\begin{lemm}
			Let $\displaystyle\lim_{x \to x_0}{f(x) = L}$ be given. Then, $\displaystyle\lim_{x \to x_0}{(kf)(x) = kL}, \forall k \in \R$.
		\end{lemm}
	\subsection{Theorem 1.4.1 (Linearity)}
		\begin{theor}
			Let $D \subseteq \R$, $f,g: D \to \R$, $x_0, a, b, L, M \in \R$ where $x_0$ is an accumulation point of $D$. Suppose
			$\displaystyle\lim_{x \to x_0}{f(x) = L}$ and $\displaystyle\lim_{x \to x_0}{g(x) = M}$. Then,
			$\displaystyle\lim_{x \to x_0}{(af(x) + bg(x)) = aL + bM}$.\\
			\horline
			\textbf{Proof:} Let $\varepsilon > 0$ and $k \neq 0$ be given. Then, by \textbf{Lemma 1.4.2},
			\begin{center}
				\begin{math}
					\begin{aligned}
						|(kf)(x) -kL| & = |kf(x) - kL|\\
						& = |k||f(x) - L|
					\end{aligned}
				\end{math}
			\end{center}
			Since $\displaystyle\lim_{x \to x_0}{f(x) = L}$ is given, $\forall \varepsilon > 0$, in particular $\frac{\varepsilon}{|k|}$,
			$\exists \delta > 0 \ni \forall x \in D, 0 < |x-a| < \delta \implies |f(x) - L| < \frac{\varepsilon}{|k|}$.
			\begin{center}
				\begin{math}
					\begin{aligned}
						|k||f(x) - L| & < |k| \cdot \frac{\varepsilon}{|k|}\\
						& < \varepsilon
					\end{aligned}
				\end{math}
			\end{center}
			\QED
		\end{theor}

	\subsection{Lemma 1.4.4 (Sum Rule)}
		\begin{lemm}
			Let $D \subseteq \R$, $f,g:D \to \R$, and $a,L, M \in \R$. Suppose $a$ is an accumulation point of $D$,
			$\displaystyle\lim_{x \to a}{f(x) = L}$, and $\displaystyle\lim_{x \to a}{g(x) = M}$. Then,
			$\displaystyle\lim_{x \to a}{(f + g)(x) = L + M}$.\\
			\horline
			\textbf{Proof:} Recall, the triangle inequality states $$|x + y| \leq |x| + |y|$$
			So, $$(f+g)(x) - (L+M) \leq |f(x) - L| + |g(x) - M|$$
			Since $\displaystyle\lim_{x \to a}{f(x) = L} \implies \exists \delta_1 > 0 \ni \forall x \in D, 0 < |x -a| <\delta_1,
			\implies 0 < |f(x) - L| < \frac{\varepsilon}{2}$ (given that $\varepsilon > 0$). Similarly, $\displaystyle\lim_{x \to a}{g(x) = M}
			\implies \exists \delta_2 > 0 \ni \forall x \in D, 0 < | x - a| < \delta_2 \implies 0 < | g(x) - M| < \frac{\varepsilon}{2}$. Choose
			$\delta = \min(\delta_1,\delta_2) \ni 0 < |x - a| < \delta$, we have $|(f+g)(x) - (L+M)| \leq |f(x) - L| + |g(x) + M| < \frac{\varepsilon}{2v} +
			\frac{\varepsilon}{2v} = \varepsilon$ \QED

		\end{lemm}
		\newpage
	\subsection{Theorem 1.4.6 (Product Rule)}
		\begin{theor}
			Let $D \subseteq \R$ and $a,L,M \in \R$. Suppose $f,g:D \to \R$ and $a$ is an accumulation point of $D$. If $\displaystyle\lim_{x \to a}{f(x) = L}$ and
			$\displaystyle\lim_{x \to a}{g(x) = L}$, then $\displaystyle\lim_{x \to a}{(f \cdot g)(x) = L \cdot M}$.\\
			\horline
			\textbf{Proof:} let $\varepsilon > 0$. By the triangle inequality, $$|(f \cdot g)(x) - (L \cdot M)|  = |f(x) \cdot g(x) - L \cdot g(x) 
			+ L \cdot g(x) - L \cdot M|$$
			Note, $|f(x) \cdot g(x) - L \cdot g(x) + L \cdot g(x) - L \cdot M| \leq |f(x) - L||g(x)| + |L||g(x) - M|$. By the \textbf{Local Boundedness Theorem},
			$\exists k, \delta > 0 \ni |g(x)| < k$ for $0 < | x - a| < \delta$. So,
			$$|f(x) - L||g(x)| + |L||g(x) - M| < k|f(x) + L| + |L||g(x) - M|$$
			Since $\displaystyle\lim_{x \to a}{f(x) = L}$ and $\displaystyle\lim_{x \to a}{g(x) = M}$, $\exists \delta_2, \delta_3 > 0 \ni |f(x) - L| < 
			\frac{\varepsilon}{2k}$ whenever $0 < |x - a| < \delta_2$ and $|g(x) - M| < \frac{\varepsilon}{2|L| + 1}$ whenever $0 < |x - a| < \delta_2$\footnote{
			$\exists L \ni |L| = 0$}
		\end{theor}
		\begin{examp}
			\textbf{Claim:} $\displaystyle\lim_{x \to a}{x^n = a^n}$.\\
			\textbf{Proof:} Let $n \in \N, a \in \R$.\\
			\textbf{Base Case:} When $n = 1$, $\lim_{x \to a}{x^1 = \lim_{x \to a}{x} = a}$.
			\textbf{Inductive Hypothesis:} $\displaystyle\lim_{x \to a}{x^n} = a^n$.
			\textbf{Inductive Step:} Suppose the \textbf{Inductive Hypothesis} holds. Then,
			\begin{center}
				\begin{math}
					\begin{aligned}
						\displaystyle\lim_{x \to a}{x^{n+1}} & = \displaystyle\lim_{x \to a}{x^n \cdot x}\\
						& = \bigg(\displaystyle\lim_{x \to a}{x^n}\bigg)\bigg(\displaystyle\lim_{x \to a}{x}\bigg)\\
						& = a^n \cdot a\\
						& = a^{n+1}
					\end{aligned}
				\end{math}
			\end{center}
			\QED
		\end{examp}
	\newpage
	\subsection{Theorem 1.4.9 (Quotient Rule)}
		\begin{theor}
			Let $D \subseteq \R$, $f,g:D \to \R$ and $a,L,M \in \R$. Let $a$ be an accumulation point of $D$. If $M \neq 0$ and $\displaystyle\lim_{x \to a}
			{f(x) = L}$ and $\displaystyle\lim_{x \to a}{g(x) = M}$, then $\displaystyle\lim_{x \to a}{\bigg(\frac{f}{g}\bigg)(x) = \frac{L}{M}}$.\\
			\horline
			\textbf{Proof:} By exercise $1.3.15$\footnote{The proof of local positivity}
			in \textbf{From Calculus to Analysis} it is shown that $\forall x \in D$ near $a$, $g(x) > \frac{|M|}{2}$.
			Specifically, $g(x) \neq 0$ near $a$. So, $\frac{1}{g(x)} \neq 0$ near $a$. Let $\varepsilon > 0$. Observe, $|\frac{1}{g(x)} - \frac{1}{M}| = 
			|\frac{M - g(x)}{Mg(x)}| = \frac{1}{|M||g(x)|} \cdot |M - g(x)|$. Since $|g(x)| > \frac{|M|}{2}$, $\frac{1}{g(x)} < \frac{2}{|M|}$, by a 
			generalization of the local positivity theorem. Let $\delta > 0 \ni |M - g(x)| < \frac{M|^2 \varepsilon}{2}$, if $0 < |x - a| < \delta$.
			Then ,$|\frac{1}{g(x)} - \frac{1}{M}| - \frac{1}{|M||g(x)|}|M - g(x)| < \frac{1}{|M|} \cdot \frac{2}{|M|} \cdot \frac{\varepsilon |M|^2}{2} =
			\varepsilon$.
			\QED
		\end{theor}

	\subsection{Theorem 1.4.14 (Composition Rule)}
		\begin{theor}
			Let $A,B,C \subseteq \R$. Suppose $f:A \to B, g: B \to C$, $a$ is an accumulation point of $A$, $b$ is an accumulation point of $B$, and $f(x) \neq b$
			when $x \neq a$ is \emph{close} to $a$. If $\displaystyle\lim_{x \to a}{f(x) = b}$ and $\displaystyle\lim_{x \to b}{g(x) = c}$, then 
			$\displaystyle\lim_{x \to a}{(g \circ f)(x) = c}$.\\
			\horline
			\textbf{Proof:} Let $\varepsilon > 0$ be given. Since $\displaystyle\lim_{y \to b}{g(y) = c}$, $\exists \delta_1 > 0 \ni 0 < |y -b| < \delta_1 \implies
			|g(y) - c| < \varepsilon$. Similarly, since $\lim_{x \to a}{f(x) = b}, \exists \delta_2 > 0 \ni 0 < |x-a| < \delta_2 \implies |f(x) - b| < \varepsilon
			= \delta_1$.\\
			Now, let $\gamma > 0 \ni 0 < |x-a| < \delta \implies f(x) \neq b, \forall x$\footnote{Note, that $f(x) \neq b$ is assumed.}. Let $\delta := \min(\gamma,
			\delta_2) \ni 0 < |x-a|<\gamma$. So, $0 < |x-a| < \gamma \implies 0 < |f(x) - b| < \delta_1 \implies |(g \circ f)(x) - c| < \varepsilon$.
			\QED
		\end{theor}
		\newpage
	\subsection{Theorem 1.4.16 (Squeeze Theorem)}
		\begin{theor}
			Let $f,g,h:D \to \R$ and Suppose $f(x) \leq h(x) \leq g(x), \forall x \neq a$ near $a$, some accumulation point of $D$. If $\displaystyle\lim_{x \to a}
			{f(x) = L}$ and $\displaystyle\lim_{x \to a}{g(x) = L}$, then $\displaystyle\lim_{x \to a}{h(x) = L}$.\\
			\horline
			\textbf{Proof:} Let $\varepsilon>0$ be given. Let $\delta_f > 0 \ni 0 < |x-a| < \delta_f \implies |f(x) - L| < \varepsilon$. Similarly, for $\delta_g >
			0 \ni 0 < |x-a| < \delta_g \implies | g(x) - L| < \varepsilon$. Note, $$|f(x) - L| < \varepsilon \implies -\varepsilon < f(x) - L < \varepsilon$$ and
			$$|g(x) - L| < \varepsilon \implies -\varepsilon < g(x) - L < \varepsilon$$ So, $$-\varepsilon < f(x) - L \leq h(x) - L \leq g(x) - L < \varepsilon$$
			when $0 < |x - a| < \delta, \delta := \min(\delta_g, \delta_f)$. Then, $$0<|x - a| < \delta \implies -\varepsilon < f(x) - L < \varepsilon \text{ and }
			-\varepsilon < g(x) - L < \varepsilon$$ $$\implies -\varepsilon < f(x) - L \leq h(x) - L \leq g(x) - L < \varepsilon.$$ 
			\QED
		\end{theor}
	\subsection{Theorem 1.4.19}
		\begin{theor}
			Suppose $D, D_1, D_2 \subseteq \R$, $D := D_1 \cup D_2$, $f:D \to \R$, and $a$ is an accumulation point of $D_1$ and $D_2$. The following,
			\begin{enumerate}
				\item[(1)] $\displaystyle\lim_{x \to a}{f(x) = L} \implies \displaystyle\lim_{x \to a}{f\Big|_{D_1}(x) = L} \text{ and } \displaystyle\lim_{x \to a}
					{f\Big|_{D_2}(x) = L}$
				\item[(2)] $\displaystyle\lim_{x \to a}{f\Big|_{D_1}(x) - L}$ and $\displaystyle\lim_{x \to a}{f\Big|_{D_2}(x) = L} \implies \displaystyle
					\lim_{x \to a}{f(x) = L}$.
			\end{enumerate}
		\end{theor}
		\newpage
	\subsection{Corollary 1.4.20}
	\label{subsec:cor1420}
		\begin{cor} Let $D := D_1 \cup D_2, f:D \to \R$ and $a$ be an accumulation point of $D_1$ and $D_2$. If $f$ is a function,
			then $\displaystyle\lim_{x \to a}{f(x)}$ exists $\Leftrightarrow \displaystyle\lim_{x \to a}{f\Big|_{D_1}(x)} = \displaystyle\lim_{x \to a}{f\Big|_{D_2}(x)}$ both exist.
		\end{cor}
		\begin{examp}
			Let $f:\R \to \R$, $f(x) := \begin{cases} 1 & \text{if } x \in \Q\\0 & \text{if } x \not\in \Q \end{cases}$
				Let $a \in \R$. We know that $a$ is an accumulation point of $\Q$ and $\R\setminus\Q$.\footnote{By the \hyperref[subsec:theorem116]{density of rationals} and \hyperref[subsec:theorem118]{irrationals} in the reals}
			By this, note $\displaystyle\lim_{x \to a}{f\Big|_{\Q}(x) = 1}$ and $\displaystyle\lim_{x \to a}{f\Big|_{\R\subseteq\Q}(x) = 0}$.
			So, $\displaystyle\lim_{x \to a}{f(x)}$ does not exist as either of its component limits\footnote{Those above.} are not equal.
			So, this function does not converge anywhere!
	\end{examp}
	\subsection{Theorem (Convergence of Bounded Sequences)}
		\begin{theor}
			A convergent sequence is bounded\footnote{where $x$ is a sequence and $p \in \R$, convergence of a sequences is denoted $x_n \to p$}
			\footnote{additionally, the complete proof was omitted from class. I may supply the full proof at a later date. Really trying not to
			cop out here\dots}.
%			\horline
%			\textbf{proof:} suppose $x_n \to p$. so, for $\varepsilon > 0, \varepsilon :=1$, $\exists n \ni n < n \implies |x_n - p| < 1$. since, 
%			$|x_n| = |x_n - p + p| = |x_n - p| + |p| < 1 + |p|, \forall n \in n$. 
		\end{theor}
	\subsection{Theorem (Threshold of Convergent Sequences)}
		\begin{theor}
			Suppose $x_n \to p$ and $p \neq 0$. $\exists N \in \N \ni n > N \implies |x_n| > |\frac{|p|}{2}|$.
		\end{theor}
	\newpage
	\subsection{Theorem (Properties of Convergent Sequences)}
	\label{subsec:PCS}	
		\begin{theor}
			Suppose sequences $a_n \to a$, $b_n \to b$, and $a,b,k \in \R$. Then,
			\begin{enumerate}
				\item[(1)] $ka_n \to ka$.
				\item[(2)] $a_n + b_n \to a + b$.
				\item[(3)] $a_nb_n \to ab$.
				\item[(4)] If $b \neq 0$, $\frac{a_n}{b_n} \to \frac{a}{b}$.
				\item[(5)] If $a_n \leq b_n \leq c_n, \forall n$ and $a_n, c_n \to a$, then $b_n \to a$.
			\end{enumerate}
		\end{theor}
	\subsection{Theorem 2.1.1 (Continuity of Composed Functions)}
		\begin{theor}
			If $f$ is continuous at $a$ and $g$ is continuous at $b := f(a)$, then $g \circ f$ is continuous at $a$.\hfill\break
			\horline
			\textbf{Proof:} Let $\varepsilon > 0$ be given. Since $g$ is continuous at $b$, $\exists \delta_g \ni \forall y \in D, |y - b| < \delta_g
			\implies |g(y) - g(a)| < \varepsilon$. Since $f$ is continuous at $a$, $\exists \delta_f > 0 \ni \forall x \in D, |x-a| < \delta_f \implies
			|f(x) - f(a)| < \delta_g$. Note, $f(x) \in D$ and $b = f(a)$. So, $|f(x) - f(a)| < \delta_g \Leftrightarrow |y - b| < \delta_g$. Choose 
			$\delta = \delta_f$. So, $|x  - a| < \delta \implies |(g \circ f)(x) - (g \circ f)(a)| < \varepsilon$.
			\QED\hfill\break
			\horline
			\textbf{Comment:} If $a$ is an accumulation point in $D$, then continuity can be re-expressed as $$\displaystyle\lim_{x \to a}{f(x) = f(a)}$$
				and given that the limit of $f$ meets the following criteria:
				\begin{enumerate}
					\item[(1)] Given that $a$ is indeed an accumulation point of $\R$.
					\item[(2)] Given that both one-sided limits of $f$ exist.
					\item[(3)] Given that both one-sided limits of $f$ are equivalent.
				\end{enumerate}
						
		\end{theor}
\addtocontents{toc}{\setcounter{tocdepth}{3}}

	\newpage
	\subsection{Theorem 3.1.5 (Order Completeness)}
	\label{subsec:theor315}
		\begin{theor}
			Let $A \subseteq \R$ and $A \neq \emptyset$. If $A$ has an \hyperref[subsec:upperbound]{upper bound}, then $A$ has a \hyperref[subsec:supremum]{least 
			upper bound}.
		\end{theor}
	\subsection{Theorem 3.1.8 (Greatest Lower Bound Theorem)}
	\label{subsec:theor318}
		\begin{theor}
			Let $A \subseteq \R$. If $A$ has a \hyperref[subsec:infimum]{greatest lower bound}.
		\end{theor}
	\subsection{Theorem 3.2.1 (Interval Theorem)}
	\label{subsec:theor321}
		\begin{theor}
			Let $A \subseteq \R$ and $A \neq \emptyset$. Then, $A$ has the \hyperref[subsec:intermediatevalueproperty]{intermediate value property} if and only if
			it $A$ is an interval.\hfill \break
			\horline
			\textbf{Proof:} $(\Leftarrow)$ Suppose $A \subseteq \R \ni A = \emptyset$ has the intermediate value property. Let a = $\inf{A}$ and $b = \sup{A}$ such 
			that $a$ and $b$ are both finite. Then, $a$ is a lower bound and $b$ is an upper bound. So $A \subseteq [a,b]$. Let $t \in \R \ni a < t < b$. Since 
			$a < t$ and $a = \inf{A}$, $t$ is not a lower bound for $A$. Thus, $\exists x \in A \ni \inf{A} = a \leq x < t$. Similarly, $\exists y \in A \ni 
			t < y \leq b = \sup{A}$. Thus, by the intermediate value property, since $x,y \in A$ and $x < t < y$, we have $t \in A$. Therefore, $(a,b) \subseteq A$.
			\QED
		\end{theor}
	\newpage		
	\subsection{Theorem 3.3.2 (Nested Interval Theorem)}
	\label{subsec:theor332}
		\begin{theor}
			Suppose the closed intervals $([a_n,b_n])_{n \in \N}$ are nested and the sequence of lengths $(b_n - a_n)$ is null\footnote{That the length of the intervals
			approaches 0}. Then, $\exists x \in \R \ni \displaystyle\cap_{n = 1}^{\infty}{[a_n,b_n]} = \{x\}$.\hfill \break
			\horline
			\textbf{Proof:} Suppose the closed intervals $([a_n,b_n])_{n \in \N}$ are nested and $(b_n - a_n)$ is null. Then,
			$$a_i \leq a_{i+1} \leq \dots \leq a_j \leq b_j \leq \dots \leq b_{j+1} \leq b_j $$
			where $i < j, \forall i,j \in \N$. So, $a_i \leq a_j \leq b_j \leq b_i \implies a_i \leq b_j, a_i \leq a_j$. Then, $\forall m,n \in \N$, $a_m \leq b_n$. 
			Let $A:= \{a_m | m \in \N\}, B := \{b_n | n \in \N\}$. Then, $\forall a \in A$ and $\forall b \in B$, $a \leq b$.If $a:= \sup{A}$ and $b:= \inf{B}$, 
			$a \leq b$. If $a < b$, then $0 < b - a \leq b_n - a_n, \forall n \in \N$. Note, if $\forall a_n,b_n > 0$, then $(b_n -a_n) \not\to 0$. This contradicts
			$(b_n - a_n)$ being null. So, $a \not < b$ by contradiction. So, $a = \sup{A} = \inf{B} = b$.\hfill \break
			Note, $\forall n \in \N, a_n \leq a$. Similarly, $b_n \geq b, \forall n \in \N$. So, $a_n \leq a = b \leq b_n$. So, $a_n \in [a_n, b_n], \forall n \in \N$. 
			So, $\{a\} \in \displaystyle\cup_{n = 1}^{\infty}{[a_n,b_n]}$. \hfill \break
			Now, let $x \in \displaystyle\cup_{n = 1}^{\infty}{[a_n,b_n]}$ .Then, $x \in [a_n,b_n], \forall n \in \N$. So, $a_n \leq x \leq b_n$. So, $x$ is an 
			\hyperref[subsec:upperbound]{upper bound} for the set $A$ and a \hyperref[subsec:lowerbound]{lower bound} for the set $B$. So, $a \leq x \leq b$. But, 
			$a = b$. So, $x = a = b$. Thus, $x \in \{a\}$. So, $\displaystyle\cup_{n = 1}^{\infty}{[a_n,b_n]} \subseteq \{a\}$. \QED
		\end{theor}
	\subsection{Corollary 3.3.4 (Binary Nested Interval Theorem)}
	\label{subsec:cor334}
		\begin{cor}
			Let $a M b$ be real numbers. Let $a_0 := a$, $b_0 := b$. Then, $\forall n \in \N_0$, $c_n := \frac{a_n + b_n}{2}$. Now, suppose either 
			$a_{n+1} = a_n$ and $b_{n+1} = c_n$ or $a_{n+1} - c_n$ and $b_{n+1} = b_n$. Then, $\displaystyle\cap_{n = 0}^{\infty}{[a_n,b_n] = \{x\}}, \exists x 
			\in \R$.
		\end{cor}
	\subsection{Theorem 3.5.1 (Roots of an Exponential)}
	\label{subsec:theor351}
		\begin{theor}
			Let $x \in \R^{>0}$, $k \in \N$. Then, $\exists y \in \R^{>0} \ni y^k = x$. So, $x^{\frac{1}{k}} = y$.
		\end{theor}
	\subsection{Theorem 3.5.4 (Irrationality of the Square Root of Two)}
	\label{subsec:theor354}
		\begin{theor}
			Suppose $\sqrt{2}$ is rational. So, $\exists p,q \in \N \ni \sqrt{2} - \frac{p}{2}$. Suppose $p$ and $q$ have no common factors. Then, 2 = $\frac{p^2}{q^2} \implies 2q^2 = p^2$.
			So, $p^2$ and furthermore, $p$ is even. Since $p$ is even, $p = 2k, \exists k \in \N$. So, $2q^2 = (2k)^2 = 4k^2$. So, $q^2 = 2k^2$ and $q$ is even.
			So, $p$ and $q$ have a common factor, 2. But $p$ and $q$ have no common factors! So, by contradiction, $\sqrt{2}$ is irrational.\QED
		\end{theor}			
		\subsection{Lemma (Finite Union of Countable Sets)}
		\label{subsec:finiteunionofcountablesets}
			\begin{lemm}
				The union of two countable sets is a countable set.\hfill\break
				\horline
				\textbf{Proof:} Let $A$ and $B$ be countable. Let $a_1,a_2,a_3,\dots \in A$be an enumeration of $A$ and $b_1,b_2,b_3,\dots \in B$ be
				an enumeration of $B$. Note, $a_1,b_1,a_2,b_2,a_3,b_3,\dots$ is an enumeration in $A \cup B$. So, $A \cup B$ a finite union of countable sets. \QED
			\end{lemm}
	\subsection{Lemma (Infinite Union of Finite Sets)}			
	\label{subsec:infiniteunionoffinitesets}
		\begin{lemm}
			If $A_k, \forall k \in \N$ is finite, then $\displaystyle\cup_{k=1}^{\infty}{A_k}$ is countable.
		\end{lemm}

	\subsection{Theorem 4.1.5 (Infinite Union of Countable Sets)}
	\label{subsec:theor415}
		\begin{theor}
			Let $A_k := \{a_{k,1}, a_{k,2}, a_{k,3},\dots \}$ and let $B_n := \{a_{ij} | i+j = n\}$. Since, $B_n$ has $n-1$ elements, it is finite and countable.
			Consequently, $\displaystyle\cup_{k=1}^{\infty}{A_k}$ is countable. \QED
		\end{theor}
	\newpage
	\subsection{Theorem 4.1.7 (Cantor)}
	\label{subsec:theor417}
		\begin{theor}
			The interval $[0,1]$ is not countable.\footnote{Really sorry about lack of order in the theorem counting here\dots I need to create a template for propositions, which is what this statement really is}\hfill\break
			\horline
			\textbf{Proof:} Suppose $[0,1]$ is countable. Let $\{y_k | k \in \N\} := [0,1]$\footnote{Very weird notation\dots} Suppose $x := 0.d_1d_2\dots$ where
			$$d_k := \begin{cases} 7 ,& k^{\text{th}} \text{ digit of } y_k \leq 4\\ 2 ,& k^{\text{th}} \text{ digit of } y_k \geq 5 \end{cases}$$
				For any $k$, $x_k \neq y_k$ since they have different $k^{\text{th}}$ digits. \QED
		\end{theor}
	\subsection{Theorem 4.2.1}
	\label{subsec:theor421}
		\begin{theor}
			Let $A$ and $B$ be sets. There is an injective map $f:A \to B$ if and only if there is a surjective map $g: B \to A$. 
		\end{theor}
	\subsection{Theorem (Cantor-Bernstein-Schr{\"o}der Theorem)}
	\label{subsec:CBST}
		\begin{theor}
			If $|A| \geq |B|$ and $|A| \leq |B|$, then $|A| = |B|$.\footnote{If you are interested in Homotopy Type Theory, do note that \href{https://homotopytypetheory.org/2020/01/26/the-cantor-schroder-bernstein-theorem-for-\%e2\%88\%9e-groupoids/}{there exists an analog!}}
		\end{theor}
		\begin{examp}
			\textbf{Claim:} The Cantor Set is uncountable.
			\textbf{Proof:} Let the Cantor function $f: C \to [0,1]$, $f(d_0.d_1\dots) := \displaystyle\sum_{k = 0}^n{\frac{d_k}{n}}$.\hfill\break
			\textbf{Claim:} $f$ is not injective. Let $x \in [0,1]$ be expressed $x:= 0._2d_1d_2\dots$. Since $d_j \in \{0,2\}, \forall j, y \in C$ and $f(y) = x$.
			So, $f:C \to [0,1]$ is a surjection. By \hyperref[subsec:theor421]{Theorem 4.2.1}, there is an injection  $g:[0,1] \to C$.\hfill\break
			Note, $h: C \to [0,1], h(c) = c, \forall c \in C$ is a satisfactory injection as $C \subseteq [0,1]$. So, by \hyperref[subsec:CBST]{Cantor-Bernstein-Schr{\"o}der Theorem}
			$|C| = |[0,1]|$ and thus $f$ is a bijection. So, $C$ is uncountable. \QED
		\end{examp}

	\subsection{Theorem (Jump Discontinuity Theorem)}
	\label{subsec:JDT}
		\begin{theor}
			Let $I$ be an interval and suppose $ff: I \to \R$ is \hyperref[subsec:monotone]{monotone}. Let $a \in I$. Then, $f$ is discontinuous at $a$ if and only
			if $f$ has a \hyperref[subsec:jumpdiscontinuity]{jump discontinuity} at $a$.
		\end{theor}

	\subsection{Corollary 5.1.4}
	\label{subsec:cor514}
		\begin{cor}
			A monotone function has a countable or empty set of discontinuities.\hfill\break
			\horline
			\textbf{Proof:} Without loss of generality, consider $f$ to be increasing. Let $A$ be the set of discontinuities of $f$. If $a \in A$, then $\displaystyle\lim_{x \to a^{-}}{f(x) < \displaystyle\lim_{x \to a^+}{f(x)}}$ since $f$ is increasing and discontinuous at $a$. Consider the open interval,
			$$I_0 := (\displaystyle\lim_{x \to a^-}{f(x)},\displaystyle\lim_{x \to a^+}{f(x)})$$. If $\exists a^{\prime} \in A \ni a^{\prime} < a$, then
			$$\displaystyle\lim_{x \to a^-}{f(x)} \leq f(\frac{a+a^{\prime}}{2}) \leq \displaystyle\lim_{x \to a^-}{f(x)}$$\QED
		\end{cor}

	\subsection{Theorem 5.1.6 (Continuity Theorem for Monotone Functions)}
	\label{subsec:theor516}
		\begin{theor}
			A \hyperref[subsec:monotone]{monotone function} defined on an interval is continuous if and only if its range is an interval.
		\end{theor}

	\newpage		
	\subsection{Corollary 5.1.8}
	\label{subsec:cor518}
		\begin{cor}
			A monotone function defined on an interval is continuous if and only if its range has the \hyperref[subsec:intermediatevalueproperty]{intermediate value property}.
		\end{cor}

	\subsection{Corollary 5.1.9}
	\label{subsec:cor519}
		\begin{cor}
			The inverse of a strictly increasing function is strictly increasing.
		\end{cor}

	\subsection{Corollary 5.1.10}
	\label{subsec:cor5110}
		\begin{cor}
			The inverse of a strictly monotone, continuous function defined on an interval is continuous.\hfill\break
			\horline
			\textbf{Proof:} Let $I$ be an interval and suppose $f:I \to \R$ is strictly increasing (without loss of generality) and continuous. 
			By \hyperref[subsec:cor519]{Corollary 5.1.9}, $f^{-1}$ is strictly increasing. By \hyperref[subsec:theor516]{Theorem 5.1.6}, range ${f}$ is
			an interval, $J$. So, $f^{-1}:J \to I$. Therefore, $f^{-1}$ is continuous.\QED
		\end{cor}
	\newpage

	\subsection{Theorem 5.2.1 (Bolzano's Intermediate Value Theorem)}
	\label{subsec:theor521}
		\begin{theor}
			Let $I$ be an interval and $f: I \to \R$ be a continuous function. If there are points $a,b \in I \ni f(a) < 0$ and $f(b) > 0$, then there is a point
			$c \in (a,b) \ni f(c) = 0$. \hfill\break
			\horline
			\textbf{Proof:} Without loss of generality, $a < b$. Let $a_0 := a$ and $b_0 := b$. Then, $f(a_0) < 0$ and $f(b_0) > 0$. Let $c_0 := \frac{a_0+b_0}{2}$.
			If $f(c_0) = 0 \implies c = c_0$, then there exists an element in between $a$ and $b$ that equals $0$. If $f(c_0) < 0$, let $a_1 := c_0$ and 
			$b_1 := b_0$. If $f(c_0) > 0$, let $a_1 := a_0$ and $b_1 := c_0$. In either case, $f(a_1) < 0$ and $f(b_1) > 0$. So, sequences $(a_n)$ and
			$(b_n)$ with $f(a_n) < 0$ and $f(b_n) > 0, \forall n \in \N_0$. For $c_n := \frac{a_n + b_n}{2}, a_{n+1} := a_n$ and $b_{n+1} := c_n$ or
			$a_{n+1} := c_n$ and $b_{n+1} := b_n$. Thus, $([a_n,b_n])$ are nested and $\displaystyle\cup_{n = 0}^{\infty}{[a_n,b_n]} = \{c\}, \exists c \in \R$.
			Since $f(a_n) < 0, \forall n \in \N, f(c) = \displaystyle\lim_{n \to \infty}{a_n \leq 0}$. Similarly, $f(b_n) < 0 \implies f(c) = \displaystyle\lim_{n \to \infty}{b_n \geq 0 }, \forall n \in \N$. Therefore, $f(c) = 0$. \QED

		\end{theor}

	\subsection{Theorem 5.2.2 (Intermediate Value Theorem)}
	\label{subsec:theor522}
		\begin{theor}
			If a real-valued function $f$ is continuous on some interval $I$, then the image $f(I)$ has the \hyperref[subsec:intermediatevalueproperty]{intermediate value property}.\hfill\break
			\horline
			\textbf{Proof:} Let $a,b \in I \ni a<b$ and $y_0$ be in between $f(a)$ and $f(b)$. If $f(a) < f(b)$, apply the intermediate value theorem to $g(x) := f(x) - y_0$. 
			If $f(b) < f(a)$, apply the intermediate value theorem to $g(x) := y_0 - f(x)$. \QED
		\end{theor}
	\newpage		

	\subsection{Corollary 5.2.3}
	\label{subsec:cor523}
		\begin{cor}
			If a real-valued function $f$ is continuous on some interval $I$ ,then $f(I)$ is an interval.
		\end{cor}

	\subsection{Theorem 5.5.3 (Global Boundedness Theorem)}
	\label{subsec:GBT}
		\begin{theor}
			Let $I$ be a compact interval. If $f: I \to \R$ is continuous, then $f$ is bounded on $I$.\hfill\break
			\horline
			\textbf{Proof:} By the \hyperref[subsec:theor522]{intemediate value theorem}, $f(I)$ is an interval. Suppose $f$ is unbounded on $I$. Then, $\exists x \in E, 
			\forall M \in \R \ni  |f(x)| \geq M$.\hfill\break
			Bisect $I$. Since $f$ is unbounded on $I$, it is unbounded on at least one of the halves of $I$. Bisect the unbounded half of $I$. Repeat this argument
			to obtain a sequence of nested intervals $I_{n+1} \subseteq I_n, \forall n \in \N \ni f$ is unbounded on each $I_n$. By the \hyperref[subsec:theor332]{nested interval theorem}
			, $\displaystyle\cap_{n = 0}^{\infty}{I_n} = \{x_0\}$ for $\exists x_9 \in \R$. Note, $f$ is continuous on $I$ and specifically at $x_0$. So, by 
			\hyperref[subsec:theor1312]{local boundedness}, $\exists \delta > 0 \ni f$ is bounded on $(x_0 - \delta, x_0 + \delta)$. By \textbf{Example 3.3.3} in
			\textbf{From Calculus to Analysis}, $\exists n \in \N \ni I_n \subseteq (x_0 - \delta, \delta + x_0)$. So, $I_n$ is bounded by contradiction.
		\end{theor}
	\newpage	

	\subsection{Theorem 5.3.5 (Extreme Value Theorem)}
	\label{subsec:theor535}
		\begin{theor}
			Let $I$ be a compact interval. If $f: I \to \R$ is continuous on $I$, then there are $x_{\text{min}}$ and $x_{\text{max}}$\footnote{Otherwise denoted
			as $\argmin{I}$ and $\argmax{I}$ respectively.} such that $f(x_{\text{min}}(x) \leq f(x) \leq f(x_{\text{max}}(x), \forall x \in I$.\hfill\break
			\horline
			\textbf{Proof:} Let $M := \sup{f(I)}$. Since $f$is bounded, by the global boundedness theorem, $m \in \R$. Now, bisect $I$. We know that at least
			one-half of $I$ gives $\sup{f(I)} = M$. Choose this half. Repeat this argument to obtain a sequence of nested intervals. $I_n \ni \sup{f(I_n} - M$. So,
			by the \hyperref[subsec:theor332]{nested interval theorem}, $\displaystyle\cup_{n = 0}^{\infty}{I_n - \{x_0\}}$ for $\exists x_0 \in \R$. Since
			$M =  \sup{f(I)}$, $f(x) \leq M, \forall x \in I$. So, $f(x_0) \leq M$. Suppose $f(x_0) < M$. Let $\varepsilon := \frac{1}{2}\bigg(M - f(x_0)\bigg) > 0$.
			Since $f$ is continuous, $\exists \delta > 0 \ni \forall x \in I, |x - x_0| < \delta \implies |f(x) - f(x_0)| < \varepsilon = \frac{1}{2}\bigg(M - f(x_0)\bigg)$. Viz., $\exists x \in (x_0 - \delta, x_0 + \delta) \ni f(x) < f(x_0) + \varepsilon$. So, $f(x) < \frac{1}{2}M + \frac{1}{2}f(x_0) < \frac{1}{2}M + \frac{1}{2}M = M$. 
			Choose $N \ni I_n \subseteq (x_0 - \delta, x_0 + \delta)$. Then, $M = \sup{\{f(x) | x \in I_n\}} \leq \sup{\{f(x) | x \in (x_0 - \delta, x_0 + \delta)\}}
			< f(x_0) + \varepsilon = M$ by contradiction. The argument for $x_{\text{min}}$ follows similarly. \QED
		\end{theor}
	\subsection{Corollary 5.3.6}

	\label{subsec:cor536}
		\begin{cor}
			If $f:[a,b] \to \R$ is continuous and $x_{\min}, x_{\max} \in [a,b] \ni f(x_{\min}) \leq f(x) \leq f(x_{\max}), \forall x \in [a,b]$ then
			$f([a,b]) = [f(x_{\min}),f(x_{\max})$.
		\end{cor}
	\newpage		
	\subsection{Lemma 5.4.5}

	\label{subsec:lemm545}
		\begin{lemm}
			Let $f:[0,1] \to \R$. Suppose $f$ is \hyperref[subsec:uniformcontinuity]{uniformly continuous} on $I_1 : = [0,\frac{1}{2}], I_2 := [\frac{1}{2},1]$,
			and $I_3 = [\frac{1}{4},\frac{3}{4}]$. Then, $f$ is uniformly continuous on $[0,1]$.\hfill\break
			\horline
			\textbf{Proof:} Let $\varepsilon > 0$ be given. For each $j$, let $\exists \delta_j > 0$ such that uniform continuity is satisfied for $\varepsilon$ on $I_j$.
			Let $\delta := \min{\delta_1,\delta_2,\delta_3,\frac{1}{2}}$. Then, if $x,y \in [0,1]$ with $|x - y| < \delta \leq \frac{1}{2}$. So, $x,y \in I, 
			\exists!j$. Thus, $|x - y| < \delta \leq \delta_j \implies |f(x) - f(y)| < \varepsilon$. \QED

		\end{lemm}

		\subsection{Theorem 5.4.6 (Uniform Continuity Theorem)}
		\label{subsec:theor546}
			\begin{theor}
				Let $f:[a,b] \to \R$ be continuous on $[a,b]$. Then, $f$ is uniformly continuous on $[a,b]$.\footnote{If $f$ is continuous on a compact interval, it is uniformly continuous}\hfill\break
				\horline
				\textbf{Proof:} Let $f:[a.b] \to \R$ be continuous on $[a,b]$. Suppose $f$ is not uniformly continuous on $[a,b]$. Then, 
				$\exists \varepsilon > 0, \forall \delta > 0, \exists x,y \in [a,b] \ni |x-y| < \delta$ and $|f(x) - f(y)| \geq \varepsilon$. 
				Let $\varepsilon_0 > 0$ be such that there is no $\delta > 0$ satisfying uniform continuity on $[a,b]$. By \hyperref[subsec:lemm545]{Lemma 5.4.5}, there is no $\delta > 0$ satisfying uniform continuity for $\varepsilon_0$ on at least one of the three subintervals ,$[a,\frac{a+b}{2}], [\frac{a+b}{2}, b],
				[\frac{3a+b}{4}, \frac{a+3b}{4}]$. Let $a_1 := a$ and $b_1:=b$. Repeat the argument so $\forall n \in \N$, $[a_n{n+1}, b_{n+1}] \subseteq
				[a_n,b_n]$ and $b_{n+1} - a_{n+1} = \frac{|b_n - a_n|}{2}$ and there is no $\delta$ satisfying uniform continuity for $\varepsilon_0$
				on $[a_n,b_n]$. Note, $[a_n,b_n]$ is nested $\forall n \in \N$. Additionally, $b_n - a_n = \frac{(b - a)}{2^{n-1}}$ is null. By the
				\hyperref[subsec:theor332]{Nested Interval Theorem}, $\exists x_0 \in \R \ni \{x_0\} = \cap_{n = 1}^{\infty}{[a_n,b_n]}$. So, $f$ is 
				continuous at $x_0$. Thus, $\exists \delta_0 > 0, \forall x \in [a,b], |x-x_0| < \delta_0 \implies |f(x) - f(x_0)| < \frac{\varepsilon_0}{2}$.
				Let $N$ be such that $[a_N,b_N] \in (x_0-\delta_0,x_0+\delta_0)$. So, $\forall x,y \in [a_N,b_N], |x - x_0| < \delta_0$ and 
				$|y-x_0| < \delta_0$. Thus, $|f(x) - f(x_0)| < \frac{\varepsilon}{2}$ and $|f(y) - f(x_0)| < \frac{\varepsilon}{2}$. So, 
				$|f(x) - f(y)| \leq |f(x_0) - f(y)| + |f(x) + f(x_0)| < \frac{\varepsilon_0}{2} + \frac{\varepsilon_0}{2} = \varepsilon_0$.

				So, $x,y \in [a_N,b_N]$ satisfy uniform continuity for $\varepsilon_0$.\footnote{This argument shows that any $\delta>0$ satisfies uniform
				continuity for $\varepsilon$ on $[a_N, b_N]$. But, there is supposedly no $\delta_0$ that satisfies uniform continuity for $\varepsilon_0$
				on $[a_N, b_N]$} So. $f$ is uniformly continuous on $[a,b]$.\QED 
			\end{theor}

		\subsection{Theorem (Product Rule)}
		\label{subsec:productrule}
			\begin{theor}
				Suppose $a$ is an accumulation point of $D$ and $f,g: D \to \R$ are differentiable at $a$. Then, 
				$$(fg)^{\prime}(a) = f^{\prime}(a)g(a) + f(a)g^{\prime}(a)$$
			\end{theor}

		\subsection{Theorem (Chain Rule)}
		\label{subsec:chainrule}
			\begin{theor}
				Suppose $g: A \to B$ is differentiable at $a$ and $f:B \to C$ is differentiable at $g(a)$. Then, $(f \circ g)$ is differentiable at $a$ and 
				$$(f \circ g)^{\prime})a_ = f^{\prime}(g(a))g^{\prime}(a)$$
			\end{theor}

		\subsection{Theorem (Quotient Rule)}
		\label{subsec:quotientrule}
			\begin{theor}
				Suppose $a \in D$ is an accumulation point and $f,g:D \to \R$ are differentiable at $a$ with $g(a) \neq 0$. Then, $\frac{f}{g}$ is 
				differentiable at $a$ and $$\big(\frac{f}{g}\big) = \frac{f^{\prime}(a)g(a) - f(a)g^{\prime}(a)}{(g(a))^2}$$
			\end{theor}

		\subsection{Lemma 7.1.1}
		\label{subsec:lemm711}
			\begin{lemm}
				Let $f:[a,b] \to \R$ be a bounded function. The set of lower sums for $f$ is non-empty and bounded from above.
			\end{lemm}

		\subsection{Lemma 7.1.8}
		\label{subsec:lemm718}
			\begin{lemm}
				Let $g$ be a step function. If $g^{\prime}$ is obtained from $g$ by inserting a finite number of additional partition points, then
				$$\sum g^{\prime} = \sum g$$
				\horline
				\textbf{Proof:} Let $g:= \sum_{k = 1}^n{A_k \mathbbm{1}_{(x_{k-1}, x_k)}}$ be a step function. Let $t \in [a,b] \ni t \neq x_i, \forall i$.\footnote{We are picking 
				a new partition point} For $k < j$, let $y_k := x_k$ and a new set partition $B_k := A_k$. If $k = j$, then let $y_j = t$ and $B_j := A_j$. Finally, for 
				$k > j$, let $y_k = x_{k-1}$ and $B_k := A_{k-1}$. Now, define $g_1 := \sum_{k=1}^{n+1}B_k \mathbbm{1}_{(y_{k-1}, y_k)}$. Then, if $x \in (x_{i-1}, x_i)$ for $i \neq j$, 
				$g_1(x) := A_i$. If $x \in (x_{j - i}, t)$ or $(t, x_j)$, then $g_1(x) = A_i$. So, 
				$$\sum_{P}{g_1} = \sum_{i = 1}^{n+1}{B_i(y_i - y{i-1})} =  \sum_{i = 1}^{n+1}{A_i(x_i - x{i-1})}$$
				\QED
			\end{lemm}

		\subsection{Corollary 7.1.9}
		\label{subsec:cor719}
			\begin{cor}
				If $s,t$ are step functions with respect to partitions $P,Q$ of $[a,b]$, then $s,t$ are step functions with respect to the \emph{common partition} $R := P \cup Q$. 
				Also, $\sum_R s = \sum_P s$ and $\sum_R t = \sum_Q t$.
			\end{cor}

		\subsection{Lemma 7.1.11}
		\label{subsec:lemm7111}
			\begin{lemm}
				Let $f$ be a bounded function. If $s$ is a lower step function for $f$ and $S$ is an upper step function for $f$, then $\sum s \leq \sum S$.\hfill\break
				\horline
				\textbf{Proof:} Suppose $s$ corresponds to $P: y_0 < \dots < y_{\ell}$ and $S$ corresponds to $Q: z_0 < \dots < z_m$. Let $\{x_0, \dots , x_n\} = \{y_0,\dots,y_{\ell}\}
				\cup \{z_0, \dots , z_m\}$. So, $R: x_0 < \dots < x_n$ is a refinement of both $P$ and $A$. By Corollary 7.1.9, $\sum_P s = \sum_R s$ and $\sum_Q S = \sum_R S$.
				Suppose $s(x) := a_i$ and $S(x) := A_i, x_{i-1} < x < x_i$. Since $s$ is a lower step function, $a_i = s(x) \leq f(x)$. Since $S$ is an upper step function,
				$f(x) = A_i$. So, $a_i \leq A_i, \forall i$\footnote{Consult Swindle. There are multiple errors in the previous edits of version 0.0.3}. Thus, 
				$\sum_R s = \sum_{i=1}^n{a_i(x_i - x_{i-1})} \leq \sum_{i=1}{A_i(x_i - x_{i-1})} = \sum_R S$.
				\QED
			\end{lemm}
		\subsection{ Lemma 7.2.1}
		\label{subsec:lemm721}
			\begin{lemm}
				If $f: [a,b] \to \R$ is a bounded function, then $\underline{\int_a^b}{f} \leq \overline{\int_a^b}{f}$.\hfill\break
				\horline
				\textbf{Proof:} Suppose $f:[a,b] \to \R$ is bounded. Let $S$ be a lower step function and $S$ an upper step function. Recall, 
				$$\underline{\int_a^b}{f} = \sup{\{\sum s | s \}}, \overline{\int_a^b}{f} = \inf{\{\sum S | S \}}$$
				Then, $\sum s \leq \underline{\int_a^b}{f}$ and $\overline{\int_a^b}{f} \leq \sum S$. By Lemma 7.1.11, $\sum{s} \leq \sum{S}$. For any $S$, $\sum{S}$ is an upper 
				bound for $\{\sum{s} | s\}$. So, $\underline{\int_a^b}{f} = \sup{ \{\sum{s} | s\}} \leq \sum{S}$. Since the above is true for all $S$, $\int_a^b{f}$ is a lower bound for 
				$\{\sum{S} | S\}$. So, $\underline{\int_a^b}{f} \leq \inf{\{\sum{s} | s \}} \leq \overline{\int_a^b}{f}$.\footnote{This lemma implies that for a bounded function $f$, 
				upper step function $S$, and lower step function $s$, $\sum{s} \leq \underline{\int_a^b}{f} \leq \overline{\int_a^b}{f} \leq \sum{S}$.}
				\QED
			\end{lemm}
		
		\subsection{Corollary 7.2.2}
		\label{subsec:cor722}
			\begin{cor}
				Suppose $f$ is integrable. If $I \in \R \ni \sum{s} \leq I \leq \sum{S}, \forall s, S$, then $\int_a^b{f} = I$.
			\end{cor}

		\subsection{Theorem 7.2.3 (Existence Theorem for Integrals)}
		\label{subsec:theor723}
			\begin{theor}
				Let $f:[a,b] \to \R$ be a bounded function. Then, $f$ is integrable if and only if given any $\varepsilon > 0$, there exists a lower step function $s_{\varepsilon}$ and 
				an upper step function $S_{\varepsilon}$ such that $$\sum{s_{\varepsilon}} - \sum{S_{\varepsilon}} < \varepsilon$$
				\horline
				\textbf{Proof:}\hfill\break
				($\Rightarrow$) Suppose $f$ is integrable. Let $\varepsilon > 0$ be given. Choose $s_{\varepsilon}$ to be a lower step function such that $\underline{\int_a^b}{f} < 
				\sum{s_{\varepsilon} + \frac{\varepsilon}{2}}$. Choose $S_{\varepsilon}$ to be an upper step function such that 
				$$\sum{S_{\varepsilon}} - \frac{\varepsilon}{2} < \overline{\int_a^b}{f}$$
				Since $\underline{\int_a^b}{f} = \overline{\int_a^b}{f}$,
				$$\sum{S_{\varepsilon} - \frac{\varepsilon}{2}} < \sum{s_{\varepsilon} + \frac{\varepsilon}{2}} \implies \sum{S_{\varepsilon}} - \sum{s_{\varepsilon}} < \varepsilon$$
				($\Leftarrow$) Now, suppose $\exists s_{\varepsilon}, S_{\varepsilon} \ni \sum{S_{\varepsilon}} - \sum{s_{\varepsilon}} < \varepsilon, \forall \varepsilon > 0$. Restated, 
				suppose $f$ is not integrable. The contrapositive is to be shown. It follows that $\underline{\int_a^b}{f} \neq \overline{\int_a^b}{f}$. Furthermore,
				$$\underline{\int_a^b}{f} < \overline{\int_a^b}{f}$$
				Let $\varepsilon := (\overline{\int_a^b}{f} - \underline{\int_a^b}{f})$. If $s_{\varepsilon}$ is a lower step function and $S_{\varepsilon}$ is an upper step function, 
				then $$\sum{s} \leq \underline{\int_a^b}{f} = \overline{\int_a^b}{f} - \varepsilon \leq \sum{S_{\varepsilon}} - \varepsilon$$
				So, $\varepsilon \leq \sum{S_{\varepsilon}} - \sum{s_{\varepsilon}}$. Thus the contrapositive has been shown.
				\QED
			\end{theor}

		\subsection{Corollary 7.2.4}
		\label{subsec:cor724}
			\begin{cor}
				Let $f:[a,b] \to \R$ be bounded. If there are lower step functions $s_n$ and upper step functions $S_n$ such that the sequence $(\sum{S_n} - \sum{s_n})$ is null. Then, 
				$f$ is integrable.
			\end{cor}

		\subsection{Theorem 7.2.5 (Evaluation Theorem for Integrals)}
		\label{subsec:theor725}
			\begin{theor}
				Let $f:[a,b] \to \R$ be a bounded function and $I \in \R$. Suppose that $\forall \varepsilon > 0$, there is a lower step function $s_{\varepsilon}$ and upper step 
				function $S_{\varepsilon}$ such that 
				$$\sum{S_{\varepsilon}} - \sum{s_{\varepsilon}} < \varepsilon \text{ and } \sum{s_{\varepsilon}} \leq I \leq \sum{S_{\varepsilon}}$$
				implies, $f$ is integrable and furthermore that $\int_a^b{f} = I$.\hfill\break
				\horline
				\textbf{Proof:} Suppose $\int_a^b{f} \neq I$. Let $\varepsilon := \frac{1}{2}|I - \int_a^b{f}|$. Then, $I$ and $\int_a^b{f}$ are both in the interval 
				$[\sum{s_{\varepsilon}}, \sum{S_{\varepsilon}}]$. Since the length of the interval is less than $\varepsilon$, $|I - \int_a^b{f}| < \varepsilon$. But, 
				$\varepsilon = \frac{1}{2}|I - \int_a^b{f}|$. So, by contradiction, $\int_a^b{f} = I$.
				\QED
			\end{theor}

		\subsection{Theorem 7.3.1}
		\label{subsec:theor731}
			\begin{theor}
				If $f:[a,b] \to \R$ is a step function, then $f$ is integrable and $\int_a^b{f} = \sum{f}$. \hfill\break
				\horline
				\textbf{Proof:} Let $s = f$, $S = f_1$, and $\varepsilon > 0$. So, $f$ is not only a step function, but is is both a lower and an upper step function. So, 
				$\sum{S} - \sum{s} = 0 < \varepsilon$. Thus, $\sum{s} \leq \sum{f} \leq \sum{S}$. 
				\QED
			\end{theor}
		\subsection{Theorem 7.3.2}
		\label{subsec:theor732}
			\begin{theor}
				Let $\varepsilon > 0$ be given. Without loss of generality, suppose $f$ is increasing. Since $f(a) \leq f(x) \leq f(b), \forall x  \in [a,b]$, $f$ is bounded. 
				Let $n \in \N \ni (f(b) - f(a)) \cdot \big(\frac{b-a}{n}\big) < \varepsilon$. Note, $(f(b) - f(a)), (b-a) > 0$. So, 
				$$\frac{\varepsilon}{(f(b) - f(a))(b-a)} > 0 \implies \exists N \in \N \ni \frac{1}{10^N} <  \frac{\varepsilon}{(f(b) - f(a))(b-a)}$$
				Define partition $x_i := a_i + \big(\frac{b-a}{n}\big)i$ for $i = 0,\dots,n$. Since $f$ is increasing, $s(x) = f(x_{i-1})$ for $x \in (x_{i-1},x_i)$ is a lower step
				function.\footnote{Since you can pick the left side and always get something smaller} Similarly, $S(x) = f(x_i)$ for $x \in (x_{i-1}, x_i)$ is an upper step function.
				Then, 
					\begin{center}
						\begin{math}
							\begin{aligned}
								\sum{S} - \sum{s} & = \sum_{i = 1}^n{f(x_i)(x_i - x_{i-1}} - \sum_{i = 1}^n{f(x_{i-1})(x_i-x_{i-1})}\\
								& = \sum_{i=1}^n{f(x_i) - f(x_{i-1})\big(\frac{b-a}{n}\big)}\\
								& = \bigg(\frac{b-a}{n}\bigg)(f(a) - f(b))\\
								& < \varepsilon
							\end{aligned}
						\end{math}
					\end{center}
				\QED

			\end{theor}
						
\newpage
\section{Version History}
	\subsection{\href{https://www.overleaf.com/read/gyqrjdfnjyvb}{v0.0.0 (4/2/2020)}}
		Initial release of \emph{The Compendium}. 		
	\subsection{v0.0.1 (4/2/2020)}
		Added \hyperref[subsec:productrule]{Product Rule}, \hyperref[subsec:chainrule]{Chain Rule}, and \hyperref[subsec:quotientrule]{Quotient Rule} to Theorems.
	\subsection{v0.0.2 (4/7/2020)}
		Added footnote about \hyperref[subsec:CBST]{The Cantor-Schr{\"o}der-Bernstein Theorem} for $\infty$-groupoids in Homotopy Type Theory.
	\subsection{v0.0.3 (4/14/2020)}
		Added \hyperref[subsec:lemm711]{Lemma 7.1.1}, \hyperref[subsec:lemm718]{Lemma 7.1.8}, \hyperref[subsec:cor719]{Corollary 7.1.9}, and \hyperref[subsec:lemm7111]{Lemma 7.1.11}.
		Additionally, the definitions for \hyperref[subsec:partition]{partitions}, \hyperref[subsec:refinement]{refinements}, \hyperref[subsec:lowerstepfunction]{lower}, 
		\hyperref[subsec:upperstepfunction]{upper}, and \hyperref[subsec:stepfunction]{general} step functions as well as \hyperref[subsec:lowerintegral]{lower} integrals, 
		\hyperref[subsec:upperintegral]{upper integrals}, and \hyperref[subsec:riemannintegrable]{Riemman integrable functions} were added. There are most likely errors in the proofs for the
		new lemmas, so I would highly recommend that someone look over them as I had some trouble understanding the previous lectures.

	\subsection{v0.0.4 (4/19/2020)}
		Added \hyperref[subsec:lemm721]{Lemma 7.2.1}, \hyperref[subsec:cor722]{Corollary 7.2.2}, \hyperref[subsec:theor723]{Theorem 7.2.3}, \hyperref[subsec:cor724]{Corollary 7.2.4}, 
		\hyperref[subsec:theor725]{the evaluation theorem for integrals}, \hyperref[subsec:theor731]{Theorem 7.3.1}, and \hyperref[subsec:theor732]{Theorem 7.3.2}.
\end{document}
